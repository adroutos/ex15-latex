\documentclass[a4paper,10pt]{article} %twocolumn
\usepackage[utf8]{inputenc} % letras acentuadas
\usepackage[portuguese]{babel} % tradução de títulos
\usepackage{algorithm} % ambiente para índice de algoritmos
\usepackage{algpseudocode} % fonte e estilo do algoritmo
%[noend]

\floatname{algorithm}{Algoritmo} % tradução da palavra algorítimo no ambiente de índice

\title{O algoritmo de conversão de AFND para ER}
\author{
    \begin{tabular}{ll}
        \textbf{Grupo:} &
        Adriano Pedro \tabularnewline &
        Eduardo Andrade\tabularnewline &
        Fernando De Abreu\tabularnewline &
        Lucas Tejo Sena\tabularnewline
    \end{tabular}
}

\begin{document}

\maketitle

\begin{abstract}

%Vamos comparar os algoritmos \textit{xsort} e \textit{ysort} para bla bla.
Mostraremos os algoritmos de conversão de \textit{afnd} para \textit{afd} e de \textit{afd} para \textit{er}.

\end{abstract}


%\section{Introdução ao Minimax}
\section{Introdução à idéia geral da conversão}

O algoritmo \textit{Minimax} trabalha percorrendo uma árvore

%\section{O Jogo \textit{Connect-4}}
\section{O algoritmo afnd $\rightarrow$ afd}

O jogo \textit{Connect-4} (C4) consiste em...

\section{Implementação}

Para conseguir blablabla

O algoritmo \textit{Minimax} segue abaixo:

\begin{algorithm}
\caption{Algoritmo Minimax}\label{alg:minimax}
\begin{algorithmic}[1]
\Function{minimax}{estado}\Comment{retorna uma ação}
\State \textbf{Entradas}: estado é a configuração atual do jogo
\State $v\gets \mathrm{maxvalor}{(estado)}$
\State \textbf{returna} a ação $a$ em sucessores(estado) cujo valor é $v$ %\Comment{comentario}
% \While{$r\not=0$}\Comment{We have the answer if r is 0}
% \State $a\gets b$
% \State $b\gets r$
% \State $r\gets a\bmod b$
% \EndWhile\label{euclidendwhile}
\EndFunction
\Function{maxvalor}{estado}\Comment{retorna o valor estático}
\If{fim(estado)}
   \State \textbf{retorna} estatico(estado)
\EndIf
\State $v \gets -\infty$
\For{todas ações $a$ nos sucessores(estado)}
    \State $v \gets \max{(v, \mathrm{minvalor}(a))}$
\EndFor
\State \textbf{retorna} $v$
\EndFunction
\Function{minvalor}{estado}\Comment{retorna o valor estático}
\If{fim(estado)}
   \State \textbf{retorna} estatico(estado)
\EndIf
\State $v \gets \infty$
\For{todas ações $a$ nos sucessores(estado)}
    \State $v \gets \min{(v, \mathrm{maxvalor}(a))}$
\EndFor
\State \textbf{retorna} $v$
\EndFunction
\end{algorithmic}
\end{algorithm}


\end{document}
